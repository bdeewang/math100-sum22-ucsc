\vspace*{1em}

\begin{definition}
Let $R$ be a relation on $A$. We call $R$ an \cdef{equivalence\ relation} if
\begin{multicols}{3}
\begin{itemize}
\item[$\bullet$] $R$ is reflexive
\item[$\bullet$] $R$ is symmetric
\item[$\bullet$] $R$ is transitive
\end{itemize}
\end{multicols}
In this case, if $aRb$, then we say $a$ is equivalent to $b$ and we alternatively denote it as $a \sim b$ (by symmetry, $b \sim a$ as well).
\end{definition}

\vspace*{1em}

An equivalence relation is a generalisation of equality.
\begin{example}
$R = \Delta_A = \setp{(a,a)}{a \in A}$ is an equivalence relation.
\[aRb\ \text{ if and only if }\ a = b \quad \text{and} \quad R = ``=''\]
Since,
\begin{itemize}
\item[(i)] $a = a$ for every $a \in A$.
\item[(ii)] If $a = b$, then $b = a$, for every $a,b \in A$.
\item[(iii)] If $a = b$ and $b = c$, then $a = c$, for every $a,b,c \in A$.
\end{itemize}
\end{example}

\vspace*{1em}

\begin{example}\label{example:eqrel1}
For $A = \set{1,2,3,4,5}$, consider the following relation on $A$
\[R = \set{(1,1),(2,2),(3,3),(4,4),(5,5),(1,3),(1,5),(5,1),(5,3),(3,1),(3,5),(2,4),(4,2)} \subseteq A \times A\]
$R$ is an equivalence relation. Note,
\begin{itemize}
\item[(i)] for every $a \in A$, we have $(a,a) \in A$.
\item[(ii)] Whenever $(a,b) \in R$, we have $(b,a) \in R$. (e.g. $(1,3) \in R$ and $(3,1) \in R$, $(3,6),(6,3) \in R$ etc.)
\item[(iii)] We have $(1,6),(6,3) \in R$, is $(1,3) \in R$? Yes! Check remaining cases.
\end{itemize}
\end{example}

\vspace*{1em}

\begin{example}\label{example:eqrel2}
Consider the following relation on $\rr^2\setminus\set{(0,0)}$
\[(x,y)R(a,b) \quad \text{if and only if (i.e. the definition)} \quad \text{there exists $\lambda \in \rr_{\neq 0}$ such that $(a,b) = (\lambda x,\lambda y)$}\]
We show $R$ is an equivalence relation.
\begin{itemize}
\item[(i)] Note that for $\lambda = 1$, we have
\[(x,y) = (\lambda x,\lambda y)\]
Therefore $(x,y)R(x,y)$, for all $(x,y) \in \rr^2$, and hence $R$ is reflexive.

\item[(ii)] Suppose $(x,y),(a,b) \in \rr^2$ is such that $(x,y)R(a,b)$. Therefore $(a,b) = (\lambda x,\lambda y)$, for some $\lambda \in \rr_{\neq 0}$, and hence $a = \lambda x,\,b = \lambda y$. Thus,
\[(x,y) = ((1/\lambda)a,(1/\lambda)b).\]
Therefore $(a,b)R(x,y)$, and hence $R$ is symmetric.

\item[(iii)] Suppose $(x,y),(a,b),(u,v) \in \rr^2$ is such that $(x,y)R(a,b)$ and $(a,b)R(u,v)$. Therefore $(a,b) = (\lambda x,\lambda y)$ and $(u,v) = (\mu a,\mu b)$, for some $\lambda,\mu \in \rr_{\neq 0}$. Hence \[a = \lambda x,\,b = \lambda y\; \text{ and }\; u = \mu a,\, v = \mu b.\] Thus,
\[(u,v) = (\mu\lambda x,\mu\lambda y).\]
Therefore $(x,y)R(u,v)$, and hence $R$ is transitive.
\end{itemize}
\end{example}

\vspace*{1em}

\begin{example}\label{example:eqrel3}
For $A = \set{\text{all people}}$, consider the following relation on $A$
\[P_1 \sim P_2 \quad \text{if and only if} \quad \text{age}(P_1) = \text{age}(P_2)\]
This is an equivalence relation. Try and verify the definition!
\end{example}

\vspace*{1em}

\begin{definition}
Let $R$ be an equivalence relation on $A$.\\[0.5em]
For $a \in A$, define the {\color{blue}$R$-}\cdef{equivalence\ class\ represented\ by} {\color{blue}$a$} to be the subset
\begin{align*}
[a] &= \setp{x \in A}{xRa \text{ (or $aRx$, by symmetry)}} = \set{\text{all elements of $A$ that are equivalent to $a$}} \subseteq A\\[0.5em]
 &= \setp{x \in A}{\text{$(x,a) \in R$ ($(a,x) \in R$)}}
\end{align*}
Since $R$ is reflexive, that is, $aRa$, we have $a \in [a]$ for any $a \in A$.
\end{definition}

\vspace*{1em}

\begin{example}
Let us find the equivalence classes for \[R = \set{(1,1),(2,2),(3,3),(4,4),(5,5),(1,3),(1,5),(5,1),(5,3),(3,1),(3,5),(2,4),(4,2)}\] described in Example \ref{example:eqrel1} on $A = \set{1,2,3,4,5}$.
\begin{align*}
[1] &= \text{the equivalence class of $1$}\\[0.5em]
 &= \text{all elements ``equivalent'' (that is, $R$-related) to $1$}\\[0.5em]
 &= \setp{x \in A}{(x,1) \in R}\\[0.5em]
 &= \set{1,3,5},\quad \text{since $(1,1),(3,1) \in R$}
\end{align*}
Similarly, 
\begin{align*}
[2] &= \setp{x \in A}{(x,2) \in R} & [4] &= \setp{x \in A}{(x,4) \in R}\\[0.5em]
 &= \set{2,4} & &= \set{2,4} = [2]\\[1em]
%
[3] &= \setp{x \in A}{(x,3) \in R} & [5] &= \setp{x \in A}{(x,5) \in R}\\[0.5em]
 &= \set{1,3,5} & &= \set{1,3,5} = [1] = [3]
\end{align*}
So, 
\begin{align*}
[1] &= [3] = [5] = \set{1,3,5}\\[0.5em]
[2] &= [4] = \set{2,4}\\
\end{align*}
We make the following observations: for $a,b \in A$,
\begin{align*}
\text{if $a\cancel{R}b$},& \text{ then $[a] \cap [b] = \emptyset$; and}\\[0.5em]
\text{if $aRb$},& \text{ then $[a] = [b]$}
\end{align*}
\end{example}

\vspace*{1em}

\begin{example}
Consider the equivalence relation as described in Example \ref{example:eqrel2}. Let's compute the equivalence class of $(3,5) \in \rr^2$
\begin{align*}
[(3,5)] &= \setp{(x,y) \in \rr^2}{(3,5) R (x,y)}\\[0.5em]
 &= \setp{(x,y) \in \rr^2}{(x,y) = (3\lambda,5\lambda),\ \text{ for some $\lambda \in \rr_{\neq 0}$}}\\[0.5em]
 &= \setp{(x,y) \in \rr^2}{x = 3\lambda\ \text{ and }\ y = 5\lambda,\ \text{ for some $\lambda \in \rr_{\neq 0}$}}\\[0.5em]
 &= \setp{(3\lambda,5\lambda) \in \rr^2}{\lambda \in \rr_{\neq 0}}
\end{align*}
This is nothing but the line defined by the equation $3y = 5x$, minus the origin.
\end{example}

\vspace*{1em}

\begin{example}
Consider the equivalence relation as described in Example \ref{example:eqrel3}. Let's compute the equivalence class of me!
\begin{align*}
[\text{Deewang}] &= \setp{P \in A}{P \sim \text{Deewang}}\\[0.5em]
 &= \setp{P \in A}{\text{age}(P) = \text{age}(\text{Deewang}) = 28}\\[0.5em]
 &= \set{\text{all people that are $28$ years of age}}
\end{align*}
\end{example}

\vspace*{1em}

\begin{example}[Congruence is an Equivalence Relation]\label{example:congeq}
Let $A = \zz$ and fix a positive integer $n$. Define a relation $R$ on $\zz$ as
\[aRb \text{ if and only if } a\equiv b \modar{n}\]
Recall that we say $a\equiv b\modar{n}$ if $n\mid (a - b)$. This is an equivalence relation.
\begin{itemize}[itemsep=1em]
\item[(i)] $a \equiv a \modar{n}$, for every $a \in \zz$.\\[0.5em]
This is because $n\mid (a-a)$ for every integer $a$, as $a - a = 0 = n\cdot 0$.

\item[(ii)] If $a \equiv b \modar{n}$, then $b \equiv a \modar{n}$, for every $a,b \in \zz$.\\[0.5em]
This is because if $a \equiv b \modar{n}$, then $n\mid (a - b)$. That is, $a - b = nk$ for an integer $k$. Therefore $b - a = n(-k)$; since $-k \in \zz$, hence $n\mid (b-a)$. Thus $b \equiv a \modar{n}$.

\item[(iii)] If $a \equiv b \modar{n}$ and $b \equiv c \modar{n}$, then $a \equiv c \modar{n}$, for every $a,b,c \in \zz$.\\[0.5em]
This is because if $a \equiv b \modar{n}$ and $b \equiv c \modar{n}$, then $n\mid (a - b)$ and $n \mid (b-c)$. That is, $a - b = nk$ and $b - c = n\ell$ for integers $k,\ell$. Therefore $a - c = (a - b) + (b - c) = n(k + \ell)$; since $k + \ell \in \zz$, hence $n\mid (a-c)$. Thus $a \equiv c \modar{n}$.
\end{itemize}
Thus the congruence relation ``$\equiv$'' is an equivalence relation.\\
\\
Let $n = 5$, we find the equivalence class of $3$ with respect to the congruence modulo $5$ relation.
\begin{align*}
[3]_5 &= \setp{n \in \zz}{n \equiv 3 \modar{5}}\\[0.5em]
 &= \setp{n \in \zz}{5\mid (n - 3)}\\[0.5em]
 &= \setp{n \in \zz}{n - 3 = 5k\text{, for some integer $k$}}\\[0.5em]
 &= \setp{n \in \zz}{n = 5k + 3\text{, for some integer $k$}}\\[0.5em]
 &= \setp{5k + 3}{k \in \zz}\\[0.5em]
 &= \set{\ldots,-7,-2,3,8,13,\ldots}\\[0.5em]
 &= \set{\text{all integers that leave remainder $3$ when divided by $5$}}
\end{align*}
\end{example}

\vspace*{1em}

\begin{example}
Let $A = \zz$, we define a relation $R$ on $\zz$ as
\[aRb \text{ if and only if } a + 3b \text{ is even}\]
We show this is an equivalence relation.
\begin{itemize}
\item[(i)] Is $aRa$, for every $a \in \zz$?\\[0.5em]
Since for any $a \in \zz$, we have that $a + 3a = 4a$ is even. Thus $aRa$, and so $R$ is reflexive.

\item[(ii)] Does $aRb$ imply $bRa$?\\[0.5em]
Suppose $aRb$, that is, $a + 3b$ is even. We show $bRa$, or $b + 3a$ is even. Note that $(a + 3b) + (b + 3a) = 4a + 4b$ is even, therefore $b + 3a = 4(a+b) - (a + 3b)$ is even, as a sum of even integers. Hence $bRa$, and thus $R$ is symmetric.

\item[(iii)] Do $aRb$ and $bRc$ imply $aRc$?\\[0.5em]
Suppose $aRb$ and $bRc$, that is, $a + 3b$ and $b + 3c$ is even. We show $aRc$, or $a + 3c$ is even. Note that $(a + 3b) + (b + 3c) = a + 3c + 4b$, therefore $a + 3c = (a + 3b) + (b + 3c) - 4b$ is even, as a sum of even integers. Hence $aRc$, and thus $R$ is transitive.
\end{itemize}
Thus, $R$ is an equivalence relation.

%\vspace*{1em}
%
%\begin{remark}
%Note that
%\begin{align*}
%aRb &\text{ if and only if } a + 3b \equiv 0 \modar{2}\\[0.5em]
% &\text{ if and only if } a \equiv -3b \modar{2}\\[0.5em]
% &\text{ if and only if } a \equiv b \modar{2} \quad \text{since $-3b \equiv b\modar{2}$, as $2\mid (-3b-b)$}
%\end{align*}
%Thus $R$ was simply the congruence relation modulo $2$ and hence is an equivalence relation. Thus,
%\begin{align*}
%[0] &= \setp{x \in \zz}{xR0}\\[0.5em]
% &= \setp{x \in \zz}{x \equiv 0 \modar{2}}\\[0.5em]
% &= \set{\text{all even integers}}\\[1em]
%%
%[1] &= \setp{x \in \zz}{xR1}\\[0.5em]
% &= \setp{x \in \zz}{x \equiv 1 \modar{2}}\\[0.5em]
% &= \set{\text{all odd integers}}
%\end{align*}
%Thus, under $R$, there are two distinct equivalence classes
%\[E = \set{\text{all even integers}},\ O = \set{\text{all odd integers}}\]
%Note: $E \cap O = \emptyset$ and $E \cup O = \zz$
%\end{remark}
\end{example}

\vspace*{2em}

\begin{mdframed}
\begin{center}
{\Large Properties of Equivalence Classes}
\end{center}
\end{mdframed}

\begin{theorem}\label{thm:eq=eqcl}
Let $R$ be an equivalence relation on a set $A$. Then, for $a, b \in A$,
\[ [a] = [b] \text{ if and only if } aRb\]
\end{theorem}
\begin{proof}
Let $a,b \in A$.
\begin{itemize}
\item[$(\Rightarrow)$] Suppose $[a] = [b]$, we show $aRb$. Note that, since $R$ is reflexive, we have $aRa$. Therefore, $a \in [a] = [b]$, and hence $a \in [b]$. Thus, $aRb$.

\item[$(\Leftarrow)$] Suppose $aRb$, we show $[a] = [b]$, that is, we show $[a] \subseteq [b]$ and $[b] \subseteq [a]$. Note that since $R$ is symmetric, we also have $bRa$.\\[0.5em]
Consider any $x \in [a]$, by definition, $xRa$. Since $xRa$ and $aRb$, by assumption, we obtain $xRb$, by transitivity. Therefore, $x \in [b]$, and hence $[a] \subseteq [b]$.\\[0.5em]
Consider any $y \in [b]$, by definition, $yRb$. Since $yRb$ and $bRa$, by assumption, we obtain $yRa$, by transitivity. Therefore, $y \in [a]$, and hence $[b] \subseteq [a]$.\\[0.5em]
Thus, $[a] = [b]$.
\end{itemize}
\end{proof}

\vspace*{1em}

We have shown that $a$ and $b$ are equivalent, if and only if $[a] = [b]$. As a consequence, we obtain the following.
\begin{theorem}
Let $R$ be an equivalence relation on a set $A$. Then, for $a, b \in A$, if $[a] \cap [b] \neq \emptyset$, then $[a] = [b]$.
\end{theorem}
Thus, given any two equivalence classes $[a]$ and $[b]$, they are either disjoint or identical.
\begin{proof}
Let $a,b \in A$, and suppose $[a] \cap [b] = \emptyset$. We show $[a] = [b]$. Consider any $x \in [a] \cap [b]$. Then $x \in [a]$ and $x \in [b]$, and therefore $xRa$ and $xRb$. Since $R$ is symmetric and $xRa$, we obtain $aRx$. Hence we have $aRx$ and $xRb$, and thus $aRb$ by transitivity of $R$. Therefore $[a] = [b]$, by Theorem \ref{thm:eq=eqcl}.
\end{proof}

\vspace*{1em}

\begin{remark}
Recall that a \cdef{partition} of $A$ is a collection of subsets $X_\alpha \subseteq A$, where $\alpha \in I$ for some index set (that is, for each element $\alpha \in I$, we have a subset $X_\alpha$) such that
\begin{align*}
X_\alpha \cap X_\beta = \emptyset &\text{ if } \alpha \neq \beta\\[0.5em]
\bigcup_{\alpha \in I}X_\alpha = A
\end{align*}
In this case, write $\displaystyle A = \coprod_{\alpha \in I}X_\alpha$ (``disjoint union'').
\end{remark}

\vspace*{1em}

\begin{theorem}
Let $R$ be an equivalence relation on a set $A$. Then, the collection of all $R$-equivalence classes defines a partition of $A$.\\[0.5em]
\emph{This partition is denoted $A/R$ (read as: ``$A$ mod $R$''), and is called \emph{the quotient set of $A$ by $R$.}}
\end{theorem}
\begin{proof}
The proof follows from the theorems above. We leave it to the reader to work out the details and write out a formal proof.
\end{proof}

\vspace*{1em}

Conversely,
\begin{theorem}
Let $\mathscr{S} = \set{X_\alpha}_{\alpha \in I}$ be a partition of a set $A$. Then, there exists a unique relation $R$ on $A$ such that the resulting partition by equivalence classes is the partition given
\end{theorem}
This is saying that \emph{giving an equivalence relation on $A$ is equivalent to defining a partition of $A$}.
\begin{proof}[Sketch of Proof]\renewcommand{\qed}{}
We first prove the existence of such a relation $R$ on $A$. Define $R$ as follows: say,
\[aRb \text{ if and only if } a,b \in X_\alpha \text{ for some $\alpha$}\]
for any $a,b \in A$. One checks that this $R$ is equivalent, verifying that $R$ is reflexive and symmetric is straightforward. Following is the argument for showing transitivity of $R$.\\[0.5em]
Suppose $aRb$ and $bRc$. By definition $a,b \in X_\alpha$ and $b,c \in X_\beta$. Therefore $b \in X_\alpha$ and $b \in X_\beta$, that is, $b \in X_\alpha \cap X_\beta$. Unless $X_\alpha = X_\beta$, we have shown that $X_\alpha \cap X_\beta \neq \emptyset$ contradicting the fact $\mathscr{S}$ is a partition. Hence, $X_\alpha = X_\beta$ and thus $a,c \in X_\alpha$. Therefore $aRc$.\\
\\
We still need to check following: (1) that $X_\alpha$ is an equivalence class for any $\alpha \in I$, we do this by exhibiting that for any $a \in X_\alpha$ we have $X_\alpha = [a]$; (2) This $R$ is the unique relation with this property.
\end{proof}

%\vspace*{1em}
%
%\begin{remark}
%We note that
%\[(a,b) \in R \text{ if and only if } aRb \text{ if and only if } a,b \in X_\alpha \text{ if and only if } (a,b) \in X_\alpha \times X_\alpha\]
%for some $\alpha \in I$. Therefore, as a subset $R \subseteq A \times A$, we have
%\[R = \coprod_{\alpha\in I} (X_\alpha \times X_\alpha)\]
%\end{remark}
%
\vspace*{1em}

\begin{example}
Let $A = \set{1,2,3,4,5,6}$, and consider a partition
\[\mathscr{S} = \set{\set{1,2},\set{3,4},\set{5,6}}\]
What is the equivalence relation $R$ reproducing this partition?\\
\\
Let $X_\alpha = \set{1,2},X_\beta = \set{3,4}$ and $X_\gamma = \set{5,6}$.
\begin{align*}
\text{Since $1,2 \in \set{1,2}$, we have}& \text{ $1R2$ or $(1,2) \in R \subseteq A \times A$}\\[0.5em]
\text{Since $3,4 \in \set{3,4}$, we have}& \text{ $3R4$ or $(3,4) \in R \subseteq A \times A$}\\[0.5em]
\text{Since $5,6 \in \set{5,6}$, we have}& \text{ $5R6$ or $(5,6) \in R \subseteq A \times A$}
\end{align*}
%What is the totality of $R$ as a subset of $A \times A$?
%\begin{align*}
%R &= (X_\alpha \times X_\alpha) \cup (X_\beta \times X_\beta) \cup (X_\gamma \times X_\gamma)\\[0.5em]
% &= \set{(1,2),(2,1),(1,1),(2,2)} \cup \set{(3,4),(4,3),(3,3),(4,4)} \cup \set{(5,6),(6,5),(5,5),(6,6)}\\[0.5em]
% &= \set{(1,1),(2,2),(3,3),(4,4),(1,2),(2,1),(3,4),(4,3),(5,6),(6,5)}
%\end{align*}
\end{example}

\vspace*{2em}

\begin{mdframed}
\begin{center}
{\Large Congruence modulo $n$}
\end{center}
\end{mdframed}

\begin{discussion}
Fix a positive integer $n \geq 2$. Recall that we say $a \equiv b \modar{n}$ if 
\[n\mid (a-b) \text{ if and only if $a$ and $b$ have the same remainder when divided by $n$}\]
We have seen in Example \ref{example:congeq} that \emph{congruence modulo $n$} is an equivalence relation. What are the equivalence classes?
\begin{align*}
[0] &= \setp{x \in \zz}{x \equiv 0 \modar{n}} = \setp{nk}{k \in \zz}\\[0.5em]
[1] &= \setp{x \in \zz}{x \equiv 1 \modar{n}} = \setp{nk + 1}{k \in \zz}\\[0.5em]
	&= \qquad \vdots\\[0.5em]
[n-1] &= \setp{x \in \zz}{x \equiv n-1 \modar{n}} = \setp{nk + (n-1)}{k \in \zz}
\end{align*}
Since $n \in [0]$, therefore $[n] = [0]$.\\[0.5em]
Similarly, since $n + 1 \in [1]$, therefore $[n+1] = [1]$; so on and so forth.\\
Note that the above list is of the form $[r]$ for $0 \leq r \leq n$, all possible remainders when a number is divided by $r$. In particular, for any integer $x \in \zz$, when divided by $n$, we have a quotient $q$ and remainder $r$ such that $x = nq + r$, and thus $x \in [r]$. Hence, $[0],[1],\ldots,[n-1]$ is the complete list of equivalence classes under the congruence relation modulo $n$.\\
\\
We denote the collection of distinct equivalence classes under congruence modulo $n$ as
\[\zz/n\zz = \underbrace{\set{[0],[1],\ldots,[n-1]}}_{\text{$n$-many elements}},\]
the set of \emph{integers modulo $n$}, read as ``$\zz$ mod $n\zz$''.\\
\\
Surprisingly, $\zz/n\zz$ has properties similar to that of $\zz$. We can define addition, subtraction and multiplication for elements in $\zz/n\zz$.
\begin{itemize}[leftmargin=*]
\item[]Addition. Given two elements $[a],[b] \in \zz/n\zz$, define
\[ [a] + [b] \defeq [a+b]\]
Here $[0]$ acts as a ``zero'' element under addition. That is,
\[ [a] + [0] = [a] = [0] + [a]\]

\item[]Multiplication. Given two elements $[a],[b] \in \zz/n\zz$, define
\[ [a] \cdot [b] \defeq [ab]\]
Here $[0]$ acts as a ``unit'' element under multiplication. That is,
\[ [a] \cdot [1] = [a] = [1] \cdot [a]\]
Also note that $[a] \cdot [0] = [0] = [0] \cdot [a]$.
\end{itemize}
\end{discussion}

%\vspace*{1em}

\begin{example}
For $n = 5$, consider $\zz/5\zz = \set{[0],[1],[2],[3],[4]}$.\\[1em]
\begin{minipage}{0.5\textwidth}
\begin{center}
{\renewcommand{\arraystretch}{1.5}%
\begin{tabular}{c|ccccc}
$+$ & $[0]$ & $[1]$ & $[2]$ & $[3]$ & $[4]$\\
\hline
$[0]$ & $[0]$ & $[1]$ & $[2]$ & $[3]$ & $[4]$\\
$[1]$ & $[1]$ & $[2]$ & $[3]$ & $[4]$ & $[0]$\\
$[2]$ & $[2]$ & $[3]$ & $[4]$ & $[0]$ & $[1]$\\
$[3]$ & $[3]$ & $[4]$ & $[0]$ & $[1]$ & $[2]$\\
$[4]$ & $[4]$ & $[0]$ & $[1]$ & $[2]$ & $[3]$\\
\end{tabular}
}
\end{center}
\[\text{addition table}\]
\end{minipage}\quad
\begin{minipage}{0.5\textwidth}
\begin{center}
{\renewcommand{\arraystretch}{1.5}%
\begin{tabular}{c|ccccc}
$\times$ & $[0]$ & $[1]$ & $[2]$ & $[3]$ & $[4]$\\
\hline
$[0]$ & $[0]$ & $[0]$ & $[0]$ & $[0]$ & $[0]$\\
$[1]$ & $[0]$ & $[1]$ & $[2]$ & $[3]$ & $[4]$\\
$[2]$ & $[0]$ & $[2]$ & $[4]$ & $[1]$ & $[3]$\\
$[3]$ & $[0]$ & $[3]$ & $[1]$ & $[4]$ & $[2]$\\
$[4]$ & $[0]$ & $[4]$ & $[3]$ & $[2]$ & $[1]$\\
\end{tabular}
}
\end{center}
\[\text{multiplication table}\]
\end{minipage}
\begin{proof}[Observation]\renewcommand{\qed}{}
In the multiplication table, we have
\[ [2]\cdot [3] = [1] = [3]\cdot [2] \quad \text{and} \quad [4]\cdot[4] = [1];\]
note that $[1]$ is ``unit'' element under multiplication.\newpage
In this way, we say
\begin{align*}
\text{$[2]$ is the ``inverse''}& \text{ of $[3]$}\\[0.5em]
\text{$[3]$ is the ``inverse''}& \text{ of $[2]$}\\[0.5em]
\text{$[4]$ is the ``inverse''}& \text{ of $[4]$}
\end{align*}
In this way, ``division'' in $\zz/5\zz$ is possible. ``Dividing'' by $[2]$ means multiplying by $[3]$, for example. 
\end{proof}
\end{example}

\vspace*{1em}

Division is not always possible for all $\zz/n\zz$.
\begin{example}
For $n = 4$, consider $\zz/4\zz = \set{[0],[1],[2],[3]}$.\\[1em]
\begin{center}
{\renewcommand{\arraystretch}{1.5}%
\begin{tabular}{c|cccc}
$\times$ & $[0]$ & $[1]$ & $[2]$ & $[3]$\\
\hline
$[0]$ & $[0]$ & $[0]$ & $[0]$ & $[0]$\\
$[1]$ & $[0]$ & $[1]$ & $[2]$ & $[3]$\\
$[2]$ & $[0]$ & $[2]$ & $[0]$ & $[2]$\\
$[3]$ & $[0]$ & $[3]$ & $[2]$ & $[1]$
\end{tabular}
}
\end{center}
\[\text{multiplication table}\]
For no $[a] \in \zz/4\zz$ do we get $[2]\cdot [a] = [1]$. So, ``dividing'' by $[2]$ has no meaning in $\zz/4\zz$. Thus, division is not possible in $\zz/4\zz$.
\end{example}

\vspace*{1em}

\begin{discussion}[Fact]
For a prime number $p,\ \zz/p\zz$ admits division.
\item[] If $n$ is not prime, you cannot do division in $\zz/n\zz$. If $n = ab,\ 1 < a,b < n$, then 
\[ [a] \cdot [b] = [n] = [0]\]
So, $[a],[b]$ are ``zero divisors''.\\
\\
In $\zz/p\zz$, where $p$ is a prime number, we can do all arithmetic operations: addition, subtraction, multiplication and division; this is an example of a \emph{field}.
\[\underbrace{\underbrace{\underbrace{\text{addition},\ \text{subtraction}}_{\text{group}},\ \text{multiplication}}_{\text{ring}},\ \text{division}}_{\text{field}}\]
\end{discussion}